%!TEX root = ../template.tex
%%%%%%%%%%%%%%%%%%%%%%%%%%%%%%%%%%%%%%%%%%%%%%%%%%%%%%%%%%%%%%%%%%%
%% chapter1.tex
%% NOVA thesis document file
%%
%% Chapter with introduction
%%%%%%%%%%%%%%%%%%%%%%%%%%%%%%%%%%%%%%%%%%%%%%%%%%%%%%%%%%%%%%%%%%%

\typeout{NT FILE chapter1.tex}%

\chapter{Introduction}
\label{cha:introduction}

\prependtographicspath{{Chapters/Figures/Covers/}}

% epigraph configuration
%\epigraphfontsize{\small\itshape}
%\setlength\epigraphwidth{12.5cm}
%\setlength\epigraphrule{0pt}
%
%\includegraphics[width=0.1\linewidth]{NOVAthesisFiles/Images/novathesis-insignia}\hfil%l
%\includegraphics[width=0.875\linewidth]{NOVAthesisFiles/Images/novathesis-text}
%
%\noindent This is the \gls{novathesis} \LaTeX\ template Version \novathesisversion\ %from \novathesisdate.
%
%\epigraph{
%  This work is licensed under the %\href{https://www.latex-project.org/lppl/lppl-1-3c/}{\LaTeX\ Project Public License %v1.3c}.
%  To view a copy of this license, visit the %\href{https://www.latex-project.org/lppl/}{LaTeX project public license}.
%}

\section{Context} 
\label{sub:intro-context}

Nowadays, the use of Decentralized Systems is a must for applications of considerable size. A Decentralized System is a system composed by multiple machines that interact with each other. This architecture allows the development of applications that do not have a single point of failure, which means that even if a machine fails, the system can continue to operate properly, thus providing high availability. Another advantage of these systems is that it is possible to combine the resources from multiple machines and possibly reduce the waiting time required to execute operations. For the context of this dissertation we first introduce the following related aspects: context of a DL, blockchains and their data structures, consistency and consensus in DLs, permissionless DLs and security issues, consensus models and types.

\textbf{Distributed Ledgers (DLs).} A \gls{DL} is a decentralized system that must maintain shared state among a large number of nodes in physically separated machines, where no trusted central authorities coordinate the process. In the context of the dissertation, it is particularly interesting to consider blockchain platforms and their challenges to achieve decentralization, consistency and security requirements while maintaining good performance in large scale settings.

\textbf{Blockchains and data structures.} A blockchain is a data structure composed by an ordered \gls{DL} of blocks, in which a block contains a set of transactions. This architecture requires to solve the problem of achieving a common state in Distributed Systems, in the sense that every node must have the same ordered ledger. In fact, blockchains are a very active research field \cite{Blockchain_Consensus_Algorithms-A_Survey, the_blockchain_state_of_the_art} and one of its major topics is the property of achieving decentralization through the use of consensus algorithms. % Some classical solutions have already been studied \cite{rbft, pbft, zyzzyva_speculative_bft, byzantine_generals_problem, fast_bft} but can only be applied for permissioned (private) Blockchains. This type of Blockchains are normally smaller in the number of participants where all users must be registered with some kind of access control in a centralized authority.

\textbf{Consistency and consensus for DLs.} Consistency and replication requirements for \gls{DL}s are strongly related to the solution for the consensus problem. Consensus solutions in the classical permissioned setting have been extensively studied in distributed systems literature \cite{rbft, pbft, zyzzyva_speculative_bft, byzantine_generals_problem, fast_bft} and are based on the fact that nodes have well-known identities.

\textbf{Permissionless ledgers and security issues.} In the case of permissionless ledger systems, consensus models must address the required safety and liveness guarantees for the ledger consistency, with different challenges compared with the permissioned model, namely scalability. These challenges are closely related in exploring mechanisms to optimize tradeoffs imposed by the employed consensus mechanisms, such as: throughput, latency in block-finalization, etc. In these platforms, participants are able to read, write, verify and eventually enter in the consensus protocol leading to structural challenges for building blockchains of this type because the solutions presented above cannot be applied in this context due to the following issues \cite{Rethinking_Large-Scale_Consensus}: unauthenticated communication (all participants are anonymous), scaling (the protocol participants may be uncertain about the exact number of active players in the protocol), malicious (byzantine) behaviour from some nodes and dynamic membership (nodes can enter or exit the system at any time).

\textbf{Consensus models and types.} Nevertheless, Bitcoin \cite{bitcoin} was the first application to successfully address all the issues inherent with it, namely the \textit{Double Spending} problem which states that a digital currency cannot be reused in two transactions at the same time. In order to achieve that result, Bitcoin implements a consensus algorithm based on a \gls{PoW} \cite{pow} approach, in which all participating nodes compete for the right of writing a block into the \gls{DL} by comparing computing power. Although this solution solves the issues presented above in a relatively fair and decentralized way, there are some tradeoffs that limit the wider adoption of this system, namely throughput, consistency and lots of wasted computing power on useless hash calculations. In fact, the throughput is close to 7 transactions/sec and is much lower when comparing with the traditional payment system \textit{Visa} that can handle a peak rate of 56 000 transactions/sec \cite{pow_vs_visa}. % The another one is consistency, in the sense that it is possible for mined blocks already in the blockchain to be reverted and discarded due to forks on the chain. The last one is inherited 

To reduce the amount of consumed energy and the time required for consensus in \gls{PoW}, \gls{PoS} \cite{pow_vs_pos_eval_performance_and_security} introduces the concept of stake related with the right for a node to add a block to the chain. However, the tradeoffs are the lack of decentralization of the base of trust which increases the possibility of the system to be attacked.

On the other hand, \gls{PoET} \cite{poet_security_analysis} is a consensus algorithm used by the Hyperledger Sawtooth \cite{hyperledger_sawtooth} similar to \gls{PoW} in the sense that it relies on the concept of electing a leader in each round to propose the next block to the distributed ledger. Comparing with \gls{PoW}, this algorithm is much more efficient because it only relies on the Intel \gls{SGX} capability which acts as a lottery mechanism to elect the leader, thus do not need to solve a computationally intensive cryptographic puzzle. The disadvantages is that it is necessary for a machine to have an Intel processor with \gls{SGX} capability.

Unlike the above algorithms, \gls{PBFT} \cite{pbft} is a fast mechanism at achieving consensus and solves all the above tradeoffs at the price of limiting scalability because it requires more than 3 times the number of byzantine nodes in the system to operate properly. \gls{PBFT} consensus was initially adopted in the design of permissioned blockchains \cite{hyperleder_fabric, r3-corda, tendermint}. However, due to its advantages, some permissionless blockchains are employing this type of algorithm by using a hybrid solution \cite{bitcoin-ng, hybrid_consensus, peercensus, byzcoin, solida}. We will present the above consensus models in more detail in chapter \ref{cha:background}.


% This algorithm solves all the above tradeoffs (throughput, consistency, useless computations and decentralization) but cannot scale because it requires the total number of nodes to be larger than 3 times the number of byzantine ones.

% However, some solutions are being created to try to mitigate the mentioned tradeoffs and allow for an application to choose a specific type of consensus that better suits its requirements. With this approach, the structure of components for Blockchains that follows the requirements of applications is the current state-of-the-art of this technology. 

% Atualmente, a tecnologia de \textit{Blockchain} \cite{the_blockchain_state_of_the_art}, ou também designada por \textit{Ledger} Distribuído, é uma área recente de investigação que permite o desenvolvimento de aplicações distribuídas de grande dimensão. Pelo facto destas aplicações serem distribuídas estão inerentes certas propriedades, tais como: disponibilidade, tolerância a falhas, aproveitamento dos recursos de múltiplas máquinas, entre outras. Isto acontece pois, devido ao elevado grau de distribuição por várias máquinas, não existe um único ponto de falha, permitindo recuperar/mascarar falhas, tendo como consequência o aumento da disponibilidade do sistema. Esta aproximação também permite utilizar totalmente a capacidade computacional conjunta de múltiplas máquinas, melhorando a eficiência das computações e reduzindo o tempo de espera. Deve-se também ter em atenção que os dados guardados numa aplicação são a parte central das mesmas, os quais devem ser protegidos contra eventuais ataques e perdas.

% Desta forma, a utilização de uma \textit{Blockchain} permite garantir a imutabilidade dos dados, ou seja, é assegurada a integridade dos mesmos e também a sua disponibilidade, devido à sua replicação entre diferentes máquinas. Para garantir estas propriedades, uma \textit{Blockchain} é uma estrutura de dados constituída por blocos ordenados cronologicamente, em que cada bloco contém um apontador para o anterior, formando uma cadeia de blocos imutável. Por sua vez, um bloco é constituído por um conjunto de transações, as quais contêm os dados guardados. De forma a assegurar a consistência dos blocos (que por sua vez garante a consistência dos dados) entre os diferentes participantes da \textit{Blockchain}, é utilizado um algoritmo de consenso. Estes algoritmos são a parte central deste tipo de sistemas distribuídos, pois garantem uma coordenação entre os diferentes participantes com o objetivo conduzir o sistema a um estado consistente.

% Estes algoritmos têm como objetivo conduzir o sistema a um estado consistente (em que todos os participantes possuem o mesmo estado) ao resolverem 2 problemas conhecidos, \textbf{Problema dos Generais Bizantinos} e \textbf{\textit{Double Spending}}. O \textbf{Problema dos Generais Bizantinos} insere-se no contexto em que os dados são transmitidos entre múltiplos nós através de um sistema de comunicação entre pares (\textit{Peer-to-Peer} - P2P). De facto, alguns nós podem ter um comportamento bizantino e alterar os dados em curso. Assim, os nós corretos devem ser capazes de distinguir entre dados consistentes e corrompidos. Por outro lado, \textbf{\textit{Double Spending}} indica que um recurso virtual (\textit{digital currency}) não pode ser reutilizado em duas transações ao mesmo tempo e portanto, todas as transações devem ser verificadas por múltiplos nós de forma distribuída.

% Assim, existem vários algoritmos de consenso que são utilizados em diferentes contextos. O primeiro algoritmo a ser utilizado numa aplicação com base numa \textit{Blockchain} foi o sistema de criptomoeda conhecido por \textit{Bitcoin} . Este sistema utiliza um algoritmo denominado por \textbf{\textit{Proof-of-Work} (PoW)}.





\section{Motivation}
\label{sub:intro-motivation}

% In order to improve and reduce the tradeoffs previously mentioned it was proposed a solution called Pluggable Consensus 

%In order to improve and reduce the tradeoffs previously mentioned a self adaptive consensus mechanism is needed that must be able to dynamically adapt and choose a specific type of consensus based on the current state and conditions of the system. A solution known as Pluggable Consensus is the current state-of-the-art and is 

% The structure of components that play different roles in Blockchains are an important aspect when building such systems. This approach provides a way to separate concerns and focus on particular components.
The consensus algorithm is a core component in the architecture of blockchain applications that is responsible for all the nodes to agree on the order of blocks, thus achieving a common state. Although many types of consensus are available, the choice of only one type of consensus is a limitation for the development of applications. In fact, if a single consensus is used any drastic changes in the membership of the system can cause the current consensus to not meet the needs of it and not being suitable for solving complex scenarios due to the fact that all of the mentioned consensus mechanisms suffer from some of the problems: scale, efficiency, consistency and security. % This is due to the fact that all of the mentioned consensus mechanisms suffer from some of the problems: scale, efficiency, consistency and security.

In order to improve and reduce the previously mentioned tradeoffs and the limitation for an application to select a single specific type of consensus, it is necessary to design a self adaptive consensus mechanism that must be able to dynamically adapt and choose an appropriate consensus algorithm according to the type of blockchain, user needs and the current state and conditions of the system. With this type of mechanism, the system can ensure high efficiency and stability of the employed consensus.

Following this approach, some solutions are being researched and the current state-of-the-art presents what is called as Pluggable Consensus \cite{dynamic_reconfiguration_consensus_IoT, research_self_adaptive_consensus}. A Pluggable Consensus allows to design an architecture that must provide a single abstract interface to the consensus layer with the purpose to serve as a base for the self adaptive consensus mechanism. However, the proposed solutions have two major limitations: (1) The choice of consensus corresponds to a static setup approach. Although there can be a set of available consensus mechanisms to choose from, the choice is performed only once (before the system starts operation) and cannot be updated later. (2) The other one relates to the fact that all attempts to devise a solution to this problem can only be applied to permissioned blockchains and so it was not needed to deal with scalability issues.




\section{Problem Statement}
\label{sub:intro-problem}

% Current solutions for the self adaptive consensus mechanism have some limitations due to the choice of consensus being a static approach and performed only once and also the fact that it can only be applied to permissioned Blockchains. There are proposals that can be applied to permissionless Blockchains, namely the \gls{CUP} Model \cite{cup_model}, however these proposals require to select a sink of participators in the Blockchain and this is an open challenge.

This leads to find answers for the following research questions:

% \textit{How can we use the Pluggable Consensus approach with Unknown Participants Model to design a Self Adaptive Consensus Mechanism that must be able to dynamically adapt and choose an appropriate consensus algorithm, improving performance in Decentralized Ledgers without compromising transactions throughput, scalability, consistency and decentralization?}

\textit{How can we design a pluggable consensus approach to provide an adaptive consensus service plane for a permissionless ledger that must be able to dynamically adapt and choose an appropriate consensus algorithm, improving performance without compromising transactions throughput, scalability, consistency and decentralization?}

\textit{How to address a modular approach in designing an hybrid consensus service plane to allow a flexible way of choosing between different consensus mechanisms in a permissionless ledger?}

\textit{How to design and implement smart contracts, to allow a flexible use by applications to support different transaction types that can use alternatively or in combination different consensus mechanisms provided by the hybrid consensus service plane?}



\section{Objectives and Contributions}
\label{sub:intro-obj-contrib}

To find answers for the problem-statement questions, this dissertation will study how an hybrid consensus service plane can be designed to allow for a flexible use of different consensus mechanisms, by expressing consensus guarantees and requirements, at application-level, through the use of smart contracts. For this objective we will address the design and implementation of \mysystem: a permissionless ledger architecture supporting an hybrid and flexible consensus plane for unknown participants and anti-sybil-resistance. In concordance with the main objective, we enumerate the relevant contributions provided in this thesis:

% In this thesis we propose to design, prototype and evaluate a highly modular Self Adaptive Consensus Mechanism focused on permissionless Blockchains. The purpose of this system is to optimize the following tradeoffs: transactions throughput, scalability, consistency and decentralization. This Consensus Plane will support various consensus algorithms (\gls{PoW}, \gls{PoS}, \gls{PoET}, \gls{PBFT}, etc) and switch dynamically at runtime the employed consensus based on the current characteristics of the system and the instrumentation of events, such as: performance, finalization time of blocks, the current consensus and consistency.

% In concordance with the main objective, we enumerate the relevant contributions provided in this thesis:

\begin{itemize}
    \item Definition and specification for the \mysystem~design model and architecture providing a hybrid and flexible consensus model offering multiple consensus mechanisms;
    \item Design of smart contracts and required expressiveness to support transactions processed by the \mysystem~permissionless ledger;
    \item Development of the \mysystem~prototype as a modular approach using two base blockchains: Algorand \cite{algorand} and Blockmess \cite{blockmess};
    \item Validation and experimental evaluation of the \mysystem~implementations, considering the following aspects: 1) measurements of throughput and blocks' finalization time, comparing the benefits of the approach and analysing possible drawbacks; 2) measurement of \mysystem~reaction to shifts between consensus models expressed by the provided smart contracts; and 3) performance comparison of workloads using the different consensus models provided by the \mysystem~Hybrid consensus solution.
    % \item Design and system model for a highly modular Self Adaptive Consensus Mechanism focused on the Permissionless Model;
    % \item An implementation of the desired solution with support for multiple existing and future consensus algorithms;
    % \item Integration in the Consensus Plane and validation of the solution applied for existing/in research Permissionless Blockchains, such as: Algorand \cite{algorand}, Hyperledger Sawtooth \cite{hyperledger_sawtooth} and Blockmess;
    % \item Validation of the implemented prototype, performing different evaluations: (1) analysis of throughput of transactions, latency, and system security (given the new self adaptive consensus mechanism the system continues to guarantee all security properties), (2) verification of the tradeoffs optimization by comparing two versions of the developed prototype, one is the intended solution and the other corresponds to a version that adopts a static consensus, and (3) compare performance of the system with already implemented Blockchains.
\end{itemize}

\section{Report Outline}
\label{sub:intro-outline}

The remaining chapters of this document are organized as follows:

\begin{itemize}
    \item \textbf{Chapter 2} introduces some background concepts in a top-down approach related to decentralized ledgers and different models, smart contracts and architecture of those systems for the permissionless model with a brief explanation of different consensus mechanisms.
    \item \textbf{Chapter 3} presents relevant references that cover the different facets of related work, considering the objectives and expected contributions of this thesis.
    \item \textbf{Chapter 4} discusses an initial approach to the elaboration phase, including the system model and software architecture  considerations, implementation guidelines and an initial criteria for the expected experimental observations to validate the solution. Lastly, we present the work plan to the elaboration phase.
\end{itemize}


